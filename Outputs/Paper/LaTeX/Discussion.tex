These results are consistent with those obtained in human infants \citep{skelton_biological_2017} and adults \citep{bae_why_2015,panichello_error-correcting_2019} showing repeller points aligned with the poles of the S-cone axis (the only two repeller points identified in the macaque monkeys aligned with the poles of the S-cone axis, see the zero crossings of the positive slope in Figure 1d,e), and suggest a mechanism by which the non-uniformities emerge in CIELUV. 
The CIELUV colors were defined at threshold, whereas the color-matching experiments were conducted suprathreshold. 
The results in Figure 6c suggest that supra-threshold discrimination for colors that modulate the S-cone axis is higher than at threshold, suggestive of an amplification of subcortical S-cone signals \citep{RN655}. 

The present results are consistent with the idea that color ordering and the capacity to form color categories is innate, but not color categories themselves. 
If color categories are not innate, where do they come from? We wonder whether color categories reflect the behavioral relevance of colors in the world; the relevance of things is partially culturally determined, introducing a role for language in shaping consensus color categories. 
Given that color categories can be learned, as evident in at least one macaque in our study (Figure 5) and two macaques in another study \citep{panichello_error-correcting_2019}, the introduction of language provides a mechanism by which cultures can acquire consensus color categories. 
For example, the parts of scenes labeled as objects by human observers are more likely to be warm colored while backgrounds are more likely to be cool colored \citep{rosenthal_color_2018}, and these statistics predict universal patterns in color naming \citep{gibson_color_2017}. 
We speculate that the choice biases in humans, which include at least double the number of consensus choice biases in monkeys, likely reflect cognitive biases, and that these cognitive biases achieve consensus through shared behavioral relevance and communication \citep{RN18511,RN18514,RN18602}. 
