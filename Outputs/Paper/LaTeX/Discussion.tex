The present results are consistent with the idea that color ordering and the capacity to form color categories is innate, but not color categories themselves. 
If color categories are not innate, where do they come from? We wonder whether color categories reflect the behavioral relevance of colors in the world; the relevance of things is partially culturally determined, introducing a role for language in shaping consensus color categories. 
Given that color categories can be learned, as evident in at least one macaque in our study (Figure 5) and two macaques in another study \citep{panichello_error-correcting_2019}, the introduction of language provides a mechanism by which cultures can acquire consensus color categories. 
For example, the parts of scenes labeled as objects by human observers are more likely to be warm colored while backgrounds are more likely to be cool colored \citep{rosenthal_color_2018}, and these statistics predict universal patterns in color naming \citep{gibson_color_2017}. 
We speculate that the choice biases in humans, which include at least double the number of consensus choice biases in monkeys, likely reflect cognitive biases, and that these cognitive biases achieve consensus through shared behavioral relevance and communication \citep{RN18511,RN18514,RN18602}. 
