
\paragraph{Where do categories come from?}
% Categories --> Bias, Bias --> Categories
In the absence of language, we can infer that the shared categories we observe arise either due to innate biological factors, environmental factors such as the distribution of colors in the terrestrial environment, or a combination of the two.
The categories that we identify align well with the daylight locus (the line between the blue of the daytime sky, and the yellow of the sun, which itself closely follows the Planckian locus), and also the warm/cool object/background distinction previously identified \citep{rosenthal_color_2018}. It is plausible that what we observe is the presence of two fundamental categories - `likely to be an object of interest' and `likely to \emph{not} be an object of interest'. 
%Baysian framework: it makes sense to be biased toward common things

% Signal detection perspective: it makes sense to prioritize discrimination in more useful parts of color space

\paragraph{Different color spaces for different tasks, at different stages of the visual hierarchy}
%Panichello: attractor points provide stability of memory
% extend to other spaces? general principle?

We have derived a behaviorally-derived colorspace, based on the data collected on our task, and this colorspace seems to differ meaningfully from CIELUV. %and others?
There are multiple possibilities as to why this difference could occur.
One potential source for this difference could be that multiple different colorspaces, optimised for different tasks, could exist at multiple different levels of the visual hierarchy.
For example, it seems reasonable to say that both detection and memory are tasks that we would expect a visual system to perform, and yet color vision might require different forms to serve each of these functions.
It is possible that this type of memory task recovers the colorspace that is used for short-term memory, where it makes sense to consider the trade off between accuracy and efficiency/stability \citep{panichello_error-correcting_2019}, and that the distinction with existing colorspaces, which are derived from data on simultaneous matches represents this distinction in task demands.
Multiple levels of representation of color in the primate visual hierarchy presents an opportunity for those representations to be fruitfully different.

\paragraph{Language}
% Sapir-Whorf hypothesis



\paragraph{Saturation bias}

Non-uniformities in CIELUV may also plausibly result in our nominally iso-saturated colors actually being variably saturated. This would be a concern, as it would be a reasonable prediction that higher saturation colors would be more salient, and thus more likely to be selected as responses. We see no (or very little) bias towards higher saturation colors in a control experiment. In \autoref{fig:saturationBias} it can be seen that there are a reasonable number of errors where an animal picks a higher saturation version of the same hue (lower left quadrant). Still, it is also seen that the number of errors of the inverse type (upper right quadrant) is roughly equal in number.


\begin{figure}
%\includegraphics[width=\textwidth]{../../Figures/saturationBias.png}
\includesvg[inkscapelatex=false, width=\textwidth]{../../Figures/working/Poster_components/Saturation copy.svg}
\caption{\textbf{Saturation Bias.}
Heatmap of cues and corresponding choices. Selections along the negative diagonal correspond to correct choices. Choices along the negative diagonal in the bottom left and top right quadrants show trials on which an incorrect choice was made in such a way that the hue was correct but the higher or lower saturation versions of the cue were chosen (respectively). Note: the main diagonal is expected to be filled in at a greater extent regardless of performance level since the correct choice is shown on every trial, whereas only a subset of the incorrect choices are shown.}
\label{fig:saturationBias}
\end{figure}



\paragraph{Comparison with humans} % Panichello
