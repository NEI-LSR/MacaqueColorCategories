\paragraph{Data Analysis: Mixture Modeling}\label{para:MixtureModeling}

To assess the bias in responses for each cue, we computed the distribution of responses on trials where the monkey made an incorrect choice. For each completed trial, we calculated the error as the angular difference between the correct option and the chosen option. For each cue, we computed the number of times the monkey selected each incorrect choice, normalized by the number of times each choice color was available as a choice option for all completed trials of the given cue (this was approximately uniformly distributed). 
We then fit a Gaussian with a variable floor (\autoref{eq:GaussianEquation}) to the error distribution for each cue, using the MATLAB \lstinline{fit} function with the equation defined as \lstinline{a*exp(-(((x-b)^2)/(2*c^2)))+d}. 

% demo annotated-equation code from here: https://mirrors.concertpass.com/tex-archive/macros/latex/contrib/annotate-equations/annotate-equations.pdf

%\newpage %Sometimes the annotations don't show up, the hacky solution is to force them onto a new page

\vspace{2em} 
\begin{equation} \label{eq:GaussianEquation}
    \eqnmarkbox[purple]{p1}{a}
    \cdot
    \exp
    \frac{-(x-
    \eqnmarkbox[violet]{mu}{\mu}
    )^2}{2 
    \eqnmarkbox[blue]{sigma}{\sigma}^2}
    +
    \eqnmarkbox[gray]{d}{d}        
\end{equation}

\annotate[yshift=1em]{above,left}{p1}{height of the curve's peak}
\annotate[yshift=1em]{above}{mu}{position of the center of the peak}
\annotate[yshift=-0.75em]{below,left}{sigma}{standard deviation}
\annotate[yshift=-1em]{below}{d}{floor}
\vspace{2em} 

This fit was weighted by the number of times each choice color was an option for the given cue across all completed trials. Bias was taken as the difference between the cue and the peak of the corresponding Gaussian, for each cue color (b, Equation 1). The values, for each stimulus, are plotted as the black lines in [fig:BiasCurvesCombined]. Where this line falls closer to the center of the figure than the 0deg line, there is negative bias (which in this representation is counter-clockwise), and vice versa for values above the 0deg line. Confidence intervals were extracted from ??

\paragraph{Data Analysis: Modified Target Confusability Competition Model (TCC-c)}\label{para:TCC}
One disadvantage of the mixture model for our analysis is that we can only use it to analyze the subset of trials where the animal made an incorrect response.  This is a consequence of the 4-AFC paradigm, where across all trials, correct choices will be over-represented; there is no straightforward way to establish chance level when including trials with correct choices. In order to use the full dataset (incorrect and correct trials), we developed a generative model, based on the Target Confusability Competition (TCC) model of reference {Schurgin, 2020 \#17999}. The key elements of the TCC model are a similarity function, which determines the similarity between stimulus $s_i$ and stimulus $s_j$ through a non-linear mapping of distance to similarity, and a value of $d'$, which can be thought of as describing the amount of noise acting in the system. These two elements can be used to predict the probability that a choice of colour $s_j$ will be picked from the set of $[s_j1...s_jn ]$, on a trial where the cue is $s_i$. Our implementation of the model includes some important modifications required for our purpose.
First, our model does not assume that the underlying function is the same for each stimulus. The standard TCC model obtains a single similarity function averaged across stimuli, but it does not have to, as Schurgin et al demonstrate (see Figures 1D and Extended Data Figure 5 of ref.{Schurgin, 2020 \#17999}). Since we are most interested in the differences between the functions for the different stimuli, we allow the similarity function to be fit independently for every color. Second, our model makes no assumptions about the underlying function that determines similarity. The standard TCC model uses an exponential function with additional perceptual noise (see Figure 1F of ref. {Schurgin, 2020 \#17999}), based on observations gained from collecting data on various simultaneous judgment tasks. Our model does not do this because we expect that, if choice biases are present, they will require a differently shaped similarity function for different stimuli to be discovered. Third, we refer to fits made this way as “free” fits (and the corresponding similarity matrix as “free”), since each element of the matrix is “free” to be fit optimally, independently of all other colors. Fourth, we fit our model on single datasets at a time, whereas the standard TCC model derives the similarity functions and values for d' from independent datasets.

In fitting a free similarity matrix, noise in the system can either be represented by the value of $d'$ or by modifying the “contrast” of the similarity matrix (the relationship between the highest values and the lowest values in the matrix), since we apply no constraints on the floor or peak of the function. We therefore assume a value of $d'=1$ for the similarity matrix fits.
Since we use an AFC method, as opposed to a pseudo-continuous response space, we are able to take advantage of an alternative computational method for computing the probabilities of a particular choice being made. We use the correction factors of (McGraw and Wong 1992) (their Table 3) to estimate the probability $P(X_1>max(X_2,X_3,X_4 ))$, where $X_(1:4)$ are samples from independent normal variables, with means representing the pairwise similarity values between $s_i$ and $s_j$, and variances determined by $d'$. This decreases the runtime of our model by several orders of magnitude compared to the standard TCC method of ref. {Schurgin, 2020 \#17999}.  (See the function modelPDF in %TCC_Code_InManuscriptOrder\Model\TCCUncorrelated.m from https://osf.io/j2h65/ for comparison).

Our modification of the TCC model, which we refer to as TCC-c (TCC modified to recover choice biases), uses parameter estimation techniques to construct a similarity matrix where each cell represents the similarity between stimulus $s_i$ and stimulus $s_j$. This model, and this visualization method, allows us to assess not only the mean of the bias in the responses, but the shape of the response-bias-curve, which provides insight into the source of the biases.


\paragraph{Determining Cognitive Bias vs. Stimulus-Space Non-Uniformity using TCC-c}

For these experiments, we used a nominally perceptually uniform color space: CIELUV. This space has been derived psychophysically, with the goal of minimizing differences in perceptual non-uniformity across the space, for color differences of small magnitudes (the apparent color difference between two points in one part of the space should be equal to the apparent color difference between two points in another part of the space when that the cartesian distance between the two points in each case be the same).

However, non-uniformities within the space are thought to exist {Brainard, 2010 \#4684}; and uniformity for small color differences does not assure uniformity for larger color differences {Judd, 1970 \#17915}. Moreover, uniformity for the conditions under which the psychometric measurements from which the space was determined does not assure uniformity across all viewing conditions {Siuda-Krzywicka, 2019 \#17168}. Non-uniformities in color space could lead to choice biases that manifest in a similar way to true cognitive biases when analyzed with a mixture model (see Figure 3). Distinguishing between the alternative sources of the choice biases can be accomplished with the TCC-c model analysis. We define “true cognitive bias” as when an agent is more likely to pick choice $s_2$ as a match to cue $s_1$ than they are to select choice $s_1$ as a match to cue $s_1$, and that this behavior would not be reciprocal (they would not be more likely to pick choice $s_1$ as a match to cue $s_2$ than they are to select choice $s_2$ as a match to cue $s_2$). By this definition, this type of bias would appear as an asymmetric spread away from the negative diagonal in the similarity matrix; symmetry across the diagonal would represent reciprocity. 

Variations in the symmetric spread around the negative diagonal in the similarity matrix would represent non-uniformity of the stimulus space. In areas where the behavioral space is oversampled, one would see spread away from the negative diagonal (adjacent colors are more similar than the average). In areas that are undersampled, one would see a pinch into the negative diagonal (adjacent colors are less similar than the average). 

To estimate how much of the bias in the similarity matrix of the macaque behavior data could be attributed to non-uniformity of color space, we fit a version of the TCC-c model using a single similarity function, defined by 

(which is controlled by two parameters: λ and σ, which together control the slope of the function and the extent of the flat-top of the function at zero) and instead allow the stimuli chromaticities to float. These two parameters can be thought of as the similarity function (how similar is x to y), and the perceptual function (at what point do stimuli become indistinguishable from one-another). Unfortunately, in this parameterization of the function, the parameters are highly correlated, which makes recovery of these values via model fitting difficult. An alternative parameterization where the parameters were maximally uncorrelated would be preferable. This is akin to asking: what set of relationships between the stimuli in stimulus space can best explain the data we observe?

%$$\label{eq:SimilarityFunction}     \eqnmarkbox[purple]{explambda}{\exp(x\cdot\lambda)}     \eqnmarkbox[cyan]{convolution}{\circledast}     \eqnmarkbox[blue]{Gilbert,  \#6818}{\mathcal{N}(0,\sigma^2)}$$

Could choice biases be attributed to un-intended variation in saturation across the colors? This seems unlikely because we observed little or no bias towards higher saturation colors in a control experiment (SI FIGURE?). Still, it is also seen that the number of errors of the inverse type (upper right quadrant) is roughly equal in number.

\paragraph{Reverse-engineering a uniform color space from the macaque color-matching data }

Following the fitting of the model described above, the set of modified stimuli chromaticities can be extracted ([fig:MACBEHspace]. These can be thought of as the chromaticity values for the stimuli that we used, no longer represented in CIELUV, but now in a behaviorally derived color space instead. It is then possible to define a uniformly distributed set of colors in this new space, reparameterize them by their relationship to the original colors in the new space (e.g. new chromaticity n is 60\% of the angular distance on the path between stimulus m and stimulus p), and then determine where this new set of colors would be in CIELUV ([fig:UniformStimsInCIELUV]).