\paragraph{Psychometric functions}

Psychometric functions (see Figure 2a, SI Figure 2) were estimated with a Weibull cumulative density function: %!!!!

\begin{equation} \label{eq:Weibull}
    y=\zeta+(100-\zeta-\gamma) *\left(1-e^{-(\omega / \lambda)^k}\right) 
\end{equation}

Where animal performance (y) is a function of trial difficulty ($\omega$), computed for each trial as the angular difference between the color of the cue and the color of the foil with the closest color to the cue. 
The other parameters are the floor of the function ($\zeta$), the ceiling ($\gamma$), slope ($\lambda$), and inflection point ($k$). 
All completed trials were included in the analysis of the psychometric functions. 
The number of completed trials for each animal were: 76121 (PO); 54555 (CA); 24526 (BU); 54252 (MO). 

\paragraph{Mixture Modeling}\label{para:MixtureModeling}

Following experiments in human subjects, we analyzed the distribution of color matches made to each color using a mixture model \citep{zhang_discrete_2008,bae_why_2015}; this model assumes that the shape of the distribution is normal and provides an estimate of the width of the distribution, the offset of its peak relative to the target color, and the guess rate.
Prior work has done this analysis with a von Mises distribution \citep{zhang_discrete_2008,bae_why_2015}.
We found it simpler to implement it with a Gaussian function, $f(\theta)$, centered at each target color ($\theta$):

% demo annotated-equation code from here: https://mirrors.concertpass.com/tex-archive/macros/latex/contrib/annotate-equations/annotate-equations.pdf

%\newpage %Sometimes the annotations don't show up, the hacky solution is to force them onto a new page

\vspace{2em} 
\begin{equation} \label{eq:GaussianEquation}
    f(\theta) = 
    \eqnmarkbox[purple]{alpha}{\alpha}
    \cdot
    e^{
    -\frac{(x-
    \eqnmarkbox[violet]{mu}{\mu}
    )^2}{2 
    \eqnmarkbox[blue]{sigma}{\sigma}^2}}
    +
    \eqnmarkbox[gray]{zeta}{\zeta}        
\end{equation}

\annotate[yshift=1em]{above,left}{alpha}{height of the curve's peak}
\annotate[yshift=1em]{above}{mu}{position of the center of the peak}
\annotate[yshift=-0.75em]{below,left}{sigma}{standard deviation}
\annotate[yshift=-1em]{below}{zeta}{floor}
\vspace{2em} 

where $\alpha$ is the height of the curve’s peak; $\sigma$ is the standard deviation (width), $\mu$ is the position of the center of the peak (so the offset is $\theta$-$\mu$)%!!!!!!
, and $\zeta$ is the floor (guess rate). 
To do the analysis, we used the MATLAB fit function, which was provided with the number of times each choice color was an option for the given cue across all completed trials and produced the best-fitting function along with the 95\% CI values for those parameters (SI Figure 3; note that the tails of the distributions in SI Figure 3 reach an asymptotic floor, justifying the use of the simpler Gaussian fitting procedure). %!!!!!
The offset values for each cue were smoothed with a moving average spanning three colors (16.875o) (Figure 2bc and Figure 5a, and SI Figure 4). Negative-slope zero-crossings in which the 95\% CI exceeded the zero crossing were considered category centers (see Figure 1). %!!!!!

The width of distribution of matches varied among the colors; this was also observed by Bae et al (2015) (see their Figure 7). 
This variation implies that the assumption of uniformity of the colorspace is not valid. 
Moreover, such a non-uniformity could yield an offset in the mixture model (see Figure 3ab), so any offsets recovered by the mixture model need not require a cognitive origin.  
The alternative hypotheses for the origin of offsets recovered by the mixture model (a cognitive origin versus a stimulus-space non-uniformity), prompted us to analyze the data with a more sensitive model, the Target Confusability Competition model \citep{schurgin_psychophysical_2020}.
We recognize that in principle, the mixture model could distinguish these alternative hypotheses, but we encountered some limitations using the mixture model that were readily overcome by the TCC model. 

\paragraph{Modified Target Confusability Competition Model (TCC-v)}\label{para:TCC}

The key elements of the TCC model are a similarity function, which determines the similarity between stimulus $s_i$ and stimulus $s_j$ through a non-linear mapping of distance in colorspace to perceptual similarity, and a value of d$\prime$, which can be thought of as describing the amount of noise acting in the system. These two elements can be used to predict the probability that a choice of colour $s_j$ will be picked from the set of $\left[s_{j1}...s_{jn}\right]$, on a trial where the cue is $s_i$. 
The TCC model has been deployed with the assumption that the colorspace is perceptually uniform; this assumption is implemented as a single similarity function fit for all stimuli. 
But, as Schurgin et al. \citep{schurgin_psychophysical_2020} demonstrate the similarity function need not be fixed (see their Figures 1D and Extended Data Figure 5 of ref). 
Our implementation of the model, which we call TCC-v, permits the similarity function to vary for each cue (the “v” is for vary). 

In the standard TCC model, the similarity function uses two parameters which define a Gaussian function representing perceptual noise and an exponential function; these two functions are convolved (see Figure 1F of \citep{schurgin_psychophysical_2020}). 
We simplify the similarity function such that it is defined by a Gaussian alone. 
The cost of this is that it does not allow for a distinction between the impact of perceptual noise (where the curve flattens off approaching perceptual distances of 0) and similarity (the general shape of the function). 
In practice, we found that these parameters were highly correlated, and that reduction to a single parameter substantially reduced the computational cost of model fitting, and produced parameter estimates that were more resistant to variation in model-fitting starting values. This simplification also made it easier to modify the model; for example, to allow for the peak of the function to not be at 0 (this affords the TCC-v model the same ability as the mixture model to capture offsets).  

We created four versions of the TCC-v model: the “null model” (with only parameters for the Gaussian width of the similarity function and d-prime, and thus no allowance for bias), the “cognitive bias” model (with 64 parameters corresponding to offset values shifting the peak of the similarity function for each stimulus to higher or lower hue angles, in addition to d-prime and gaussian width), and the “stimulus-space non-uniformity model” (with 64 parameters corresponding to the relative distance between each pair of neighboring stimuli, in addition to d-prime and gaussian width). 
Finally, the “free-similarity” model, does not pre-suppose any specific similarity function; every cell in the similarity matrix is an independent parameter. 
This flexibility allows for patterns of similarity that are not captured by our hypotheses. 

In fitting the model parameters, our goal is to minimize the negative log likelihood (NLL) of the observed data. 
The NLL is computed as the sum of the negated log of the probabilities of the choices that were selected being selected. 
The probability of a particular choice being selected on a particular trial ($p_t(\text{selecting choice}_n)$) is dependent on the cue, the set of choices, the similarity function ($f(x)$), the value of d-prime (d’), and the perceptual distance between stimuli (D).

\begin{equation} \label{eq:pt}
    p_t\left(\text{selecting choice}_n \mid \text{cue},\text{choices}_{i:j}, f(x), d^{\prime}\right)
\end{equation}

Where $f(x)$ is the Gaussian equation (\autoref{eq:GaussianEquation}).

\begin{equation}
    \text{NLL} = \operatorname{sum}\left(-\left(\log \left(p_t\left(\text {selecting choice}_x\right)\right)\right)\right)
\end{equation}

Where $\text{choice}_x$ was the choice that was selected on each trial

The parameter estimates for the free parameters used by the model are iteratively updated until the model reaches a stable minimum NLL. 

% all of this needs math mode updating!!!!!!!!
The “null model” is defined by $f(x)$ (where a = 1, d = 0, μ = 0) and d’, and assumes D to be uniform. 
The “cognitive bias” model is defined by f(x) (with 64 parameters for μ, 1 parameter for σ, a = 1, d = 0) and d’, and also assumes D to be uniform. 
The “stimulus-space non-uniformity” model is defined f(x) (where a = 1, d = 0, μ = 0) but specifies for each stimulus a unique value for D (64 parameters). Note that the cognitive bias model and the stimulus-space non-uniformity have the same total number of parameters (66). 
The “free similarity model” is not defined by f(x); it is defined by d’ and the similarity matrix. 
The similarity matrix is defined by 4096 (64$^$2) parameters, one for each combination of cue and possible choice (note that the free similarity matrix is not required to be symmetric).

$p_t(\text{selecting choice}_n)$ is the probability that a sample drawn from an independent normal distribution $(X_i ~ N)$ is the highest of such samples drawn for all the choices ($i:j$) on a particular trial. 
The mean (m) of each distribution is defined by the similarity value for that cue/choice combination, multiplied by d-prime, and has a variance of 1.

\begin{multline}
    p\left(\text{selecting choice}_n\right)=p\left(X_i \sim \mathcal{N}\left(m_n \cdot d^{\prime}, 1\right)>\max \left(X_{i+1: n} \sim \mathcal{N}\left(m_n \cdot d^{\prime}, 1\right)\right)\right. \\
    \text{where} m_{i: j}=\mathrm{f}\left(\mid\right. cue _{\mathrm{i}}- choices \left._{\mathrm{i}: \mathrm{j}} \mid\right)
\end{multline}

We used an AFC paradigm as opposed to a continuous response space, so we can take advantage of an alternative computational method for estimating $p_t(\text{selecting choice}_n)$, using correction factors provided by \citet{mcgraw_common_1992} (their Table 3). 
This decreases the amount of time taken to fit the TCC-v models. 
The method for computing $p_t(\text{selecting choice}_n)$ in the original TCC model of Schurgin et al. is provided here: \verb|modelPDF| in \verb|TCC_Code_InManuscriptOrder |\verb|\Model| \verb|\TCCUncorrelated.m| from \url{https://osf.io/j2h65/}. 
The method we used is provided here: \url{https://github.com/NEI-LSR/TCC_AFC}.

When fitting the free similarity matrix model, we fix d-prime to provide a constraint on the floor and ceiling of values in the similarity matrix.
D-prime is strongly correlated with the range of the values in the similarity matrix; the range of similarity values in the similarity matrix has a maximum span of 0 to 1. 
Equivalent NLL values can be obtained either by restricting the range (e.g. 0.49 to 0.51) and a high value of d-prime (e.g. 20), or by having a larger range (e.g. the full 0 to 1) and a low value of d-prime (e.g. 0.1). 
If d’ is not fixed, the model is as likely to assume a restricted range as it is a larger range (though still bounded between 0 and 1) but will take a long time to converge. 
Fixing d-prime impacts the specific values (akin to increasing or decreasing the contrast of the similarity matrix image) but it does not impact the interpretation of the similarity matrix. 
We chose a d-prime value of 1 which is a reasonable estimate for our task \citep{schurgin_psychophysical_2020}.

\paragraph{Quantitative Model Comparison}
To compare the relative performance of models, we computed Bayesian Information Criterion (BIC) values for each model from the NLL. 
The BIC values provide a unit by which we can assess whether the differences between NLL are meaningful, penalizing for number of parameters and number of trials. 
BIC also allows us to compare models with differing numbers of parameters (though note that the “cognitive bias” and “stimulus-space non-uniformity” models have the same number of parameters).

\begin{equation}
    \text{BIC} = k\ln(n)-2\ln(\hat{L})
\end{equation}

Where $\hat{L}$ is the un-negated and exponentiated NLL, n is the number of trials, and k is the number of parameters.

To compare the relative performance of the cognitive bias and stimulus-space non-uniformity models we performed 100 bootstrap iterations of the analysis. 
Each bootstrap drew 24526 trials from the total number of completed trials for each animal (24526 was the minimum number of trials of completed trials among the 4 animals). 
Both models were then fit to each bootstrap iteration, and the BIC values were computed (see Figure 4d).  %!!!!!!

\paragraph{Reverse-engineering a uniform color space from the macaque color-matching data}

The present results imply that CIELUV is perceptually non-uniform; that is, that it samples with variable density the true underlying perceptual colorspace. 
The parameters of the stimulus-space non-uniformity model describe the relative positions of the stimuli, as determined empirically. 
After converting from relative angular distances to cartesian coordinates, these can be thought of as the hue angles for the stimuli that we used, represented now in a behaviorally derived color space which we refer to as the Macaque Uniform Color Space (MUCS) (Figure 6B). 
The inverse transformation is also possible; we can define a set of hue angles that are uniformly distributed in MUCS and convert them into CIELUV. 
For example, if we define hue angle i in MUCS, we can transform that hue angle such that it is defined by its angle relative to the two experimental stimuli on either side of it (hue angle i is at y\% of the angular distance on the path between experimental stimuli a and b). 
We can then plot MUCS hue angle i in CIELUV by plotting it at the location which is y\% along the path between experimental stimuli a and b in CIELUV. Such a set of hue angles is shown in Figure 6c. 

The MATLAB script MUCS.m will generate a user-defined number of colors sampled evenly from the behaviorally generated color space. 

\section{Author contributions}

% table? e.g. https://twitter.com/AnneEUrai/status/1361356189284581387

% 'CRediT' (https://casrai.org/credit/)

Conceptualization: BRC, DJG\newline
Data curation: DJG\newline
Formal Analysis: DJG, HMS, ALYC\newline
Funding acquisition: BRC\newline
Investigation: all authors\newline
Methodology: all authors\newline
Project administration: BRC\newline
Resources: BRC\newline
Software: DJG, HMS, ALYC\newline
Supervision: BRC\newline
Validation: DJG\newline
Visualization: all authors\newline
Writing – original draft: BRC, DJG\newline
Writing – review \& editing: BRC, DJG\newline
















