At the beginning of each testing session we positioned a mouthpiece to deliver fluid reward to the animal.
The animals were trained to perform a 4-Alternative Forced Choice (4-AFC), Delayed Match to Sample task.
Each trial was initiated when the animal fixated a small symbol at the center of the screen; trials were aborted if the animal did not maintain fixation until the fixation spot disappeared (\autoref{fig:epochs}). 
Eye position was monitored with an infrared camera (ISCAN), and fixation was defined as within a ~1 degree wide area centered on the fixation spot. 
The trial sequence was as follows. Fifty ms after initiating the trial by fixating the central spot, a 2-degree-diameter “cue” pseudo-randomly drawn from the set of 64 colors appeared for 750ms at a location within 3-degrees of the fixation spot. 
The position of the cue was consistent for a given daily recording session and could vary from [where in the visual field]. 
The cue was followed by a gray screen for a brief “memory” period of 600-900ms, after which four match options appeared at roughly the same eccentricity from the fixation spot, evenly spaced apart, in the visual hemifield opposite to the cue, with the exact angular positions of the choices varying randomly trial-to-trial. 
One choice option was always a direct match to the cue, and the other three were randomly sampled without replacement from the remaining 63 stimuli. 
After a random period from 600-900 ms, the fixation spot disappeared, instructing the monkey to direct its eyes to one of the choices. 
Reward was given only if the animal selected the choice that was identical to the cue. 
If the monkey failed to make a choice within 5 seconds, the trial aborted.
Animals made choices on average within 250 ms. 
Task difficulty was defined by the set of foil colors that accompanied the direct match for the match options: trials were considered more difficult for smaller differences between the cue color and the foil closest to the cue color. 
The experiment was controlled with custom software written in MATLAB and Psychtoolbox \citep{noauthor_nei-lsrkofiko_2022, kleiner_whats_2007}.

\paragraph{Stimuli}
Colors were defined by an equiluminant plane in CIELUV color space, and were of equal saturation within this space. 
Stimuli were discs presented on a Cambridge Research Systems Display++ screen under neutral adapting conditions.% (adapting field was xyY). 
The color of the discs varied only in hue (not luminance contrast or saturation), and were sampled from 64 equally spaced points on a circle in CIELUV space with %a white point of XXX (xy), 
a radius of 37, and %a luminance of XXX (
L* = 76.0693.%). 
These values were chosen to maximize gamut while maintaining constant saturation and luminance. 
CIELUV was chosen because CIELUV has the benefit of an associated chromaticity diagram. 
% Luminance noise was added by XXX to the extent of YYY. 

% \paragraph{4-Alternative Forced Choice: Human participants}
% Human participants were recruited via Amazon Mechanical Turk to perform an analogous version of the non-human primate 4-AFC task. 
% The purpose was simply to test the extent to which the modified paradigm, providing for matches to one of four colors, recovers similar results to those obtained using a paradigm where matches are made to a ring of all the colors.  
% To initiate a trial, participants used a mouse to adjust the location of a cursor to click on a fixation cross, after which a cue was shown to one side of the fixation cross and the cursor disappeared. 
% The cue was displayed for 750 ms. 
% After the cue was extinguished, a fixation cross was shown but the cursor remained hidden to de-incentivize mouse movement (1500 ms). 
% Four choices were then shown and the cursor reappeared, and participants made their selection by using the mouse to move the cursor to their choice and clicking the mouse. 
% The pattern of results %(SI Figure 6)
% was qualitatively similar to those in published studies using the continuous ring of colors; like the published studies \citep{bae_why_2015,panichello_error-correcting_2019}, the choice biases were best modeled with a four-category model. 
