\begin{figure}
\includesvg[inkscapelatex=false, width=\textwidth]{../../Figures/working/Old/F1_StimuliAndParadigm_v3.svg}
\caption{\textbf{4-Alternative Forced Choice (4-AFC), Delayed Match to Sample Paradigm.}
\emph{A.} Cue and choice colors in the $u^*v^*$ plane of CIELUV. Colors were defined to be equi-luminant and equi-saturated in CIELUV.
\emph{B.} Cue and choice colors in DKL colorspace.
\emph{C.} The timing and visual organization of the delayed match-to-sample task that the animals performed.
} 
\label{fig:StimuliAndParadigm}
\end{figure}

\paragraph{Stimuli} 

Stimuli were discs presented on a CRS Display++ screen.
The color of stimuli varied only in hue, and were sampled from 64 equally spaced points on a circle in CIELUV space (\autoref{fig:StimuliAndParadigm}%, \autoref*{fig:suppTest}
), with a white point of XXX (xy), a radius of 37, and a luminance of XXX (L* = 76.0693). 
These values were chosen to maximize gamut while maintaining a fixed saturation and luminance. 
The background was XXX.
% g_astrctAllParadigms{1, 1}.m_strctConversionMatrices.RGBToXYZ
% Kofiko\Paradigms\CommonFunctions\luv2rgb.m
CIELUV was used, in contrast to previous work which has used CIELAB, because CIELUV has the benefit of an associated chromaticity diagram. 
We also noted that nominally equi-saturated stimuli defined in CIELAB tended to have significant variation in apparent saturation, whereas the same in CIELUV were much closer to visually equi-saturated. 
Luminance noise was added by XXX to the extent of YYY.
% Kofiko\Paradigms\CommonFunctions\fnGenerateNoisePattern.m
% Kofiko\Paradigms\TouchForceChoiceColorCategories\fnInitializeCueTrainingTextures.m (line 144)
% Kofiko\Paradigms\TouchForceChoiceColorCategories\fnParadigmTouchForceChoiceColorCategories\Callbacks.m (line 613)

The experiment was controlled by multiple computers running `Kofiko' (a MATLAB/Psychtoolbox based software for working with monkeys %\citep{Brainard_Psychophysics_1997,Pelli_PsychoPhysics_1997}
).

\paragraph{4-Alternative Forced Choice (4-AFC): Non-human primates} \label{para:4AFC}
Non-human primates were trained on a color-matching task. 
Trials begin with fixation (50 ms) on a white cross in the center of the screen. 
A cue stimulus (colored disc) is shown to one side of the fixation cross (750 ms). 
The position of the cue is invariant throughout a daily session. 
Following cue presentation, the monkey must maintain fixation (600-900 ms) before the choice stimuli appear on the screen alongside the fixation cross (500-1000 ms). 
The choices are positioned at constant eccentricity and with equal spacing in the hemifield opposite the cue stimulus, with the exact positions of the stimuli varying randomly trial-to-trial. 
One choice is always a direct match to the cue, and the other three are randomly sampled from the remaining stimuli, without replacement.
Upon offset of the fixation cross, the animal makes a selection by saccade, and is rewarded for selecting the choice that is identical to the cue. 
Animals were head-fixed at a distance of XXX from the screen. 
Stimuli had a radius of XXX degrees of visual angle, at an eccentricity of XXX/degrees from a central fixation.
% Diameter of cue/choice discs: 2.89 degrees (100 px)
% Distance between cue and choice: 5.79 degrees (200 px - g_astrctAllParadigms{1, 1}.m_strctCurrentTrial.m_strctChoiceVars.choiceRhos)
% Diameter of “fixation window”: 1.58 degrees (50 px)
% Audrey going to re-check

\paragraph{4-Alternative Forced Choice: Human participants} 
Human participants were recruited via Amazon Mechanical Turk to perform an analogous version of the non-human primate 4-AFC task. 
Participants click on an initial fixation cross to request a trial, after which a cue is shown to one side of the fixation cross (750 ms). 
After cue offset, a fixation cross is shown and the cursor is hidden to de-incentivize mouse movement (1500 ms). 
Four choices are then shown, and participants make their selection by clicking. 

%\paragraph{Pseudo-continuous color matching task: Human participants.} All 64 stimuli were displayed in a ring at XXX eccentricity. 



%\paragraph{Color-naming task} %[DANNY - get info from Sihan?]



% Distance between cue and choice: 5.79 degrees (200 px - g_astrctAllParadigms{1, 1}.m_strctCurrentTrial.m_strctChoiceVars.choiceRhos)

