\begin{abstract}

A characteristic feature of the human mind is the use of concepts{Carey, 2009 \#18670;Tenenbaum, 2011 \#17954}. To what extent does concept formation depend on language and culture? Here we exploit color as a tool of cognitive science{Lara, 2014 \#9993;Wilken, 2004 \#18702;Hardman, 2017 \#18701;Witzel, 2018 \#16170;Schurgin, 2020 \#17999;Kim, 2021 \#18700;Maule, 2023 \#18696;Block, 2023 \#18672} and ask if macaque monkeys have a substrate for color concepts, evident as color categories. Color is continuous yet readily categorized, and for humans, color categories express meaning-laden concepts. For example, a ripe strawberry, an embarrassed child, a valentine’s heart, all derive meaning from redness. Macaques have the same spectral sensitivity as humans{Schnapf JL, 1987 \#363} and very similar central visual pathways{Lafer-Sousa, 2016 \#17582}, but for our purposes they usefully lack language. We adapted a nonverbal match-to-sample paradigm used in humans to uncover color categories from errors in matches{Bae, 2015 \#9761}; the colors are sampled from a presumed perceptually uniform space of colors{Brainard, 2010 \#4684}. With this method, macaques seemed to show two consensus color categories, not four as in humans; these corresponded to “warm” and “cool”. We next determined whether the results were better explained by simulated data modeling the impact of true cognitive categories or unwitting distortions in the color space. The results strongly favor the second explanation, providing evidence of a nonlinear mapping between a true underlying perceptually uniform representation of colors and the color space. From the empirical results, we reverse-engineered a uniform perceptual space, to our knowledge the first color similarity space unconfounded by language. The results show that human color categories are likely not innate and suggest that widely assumed “uniform” color spaces are inexorably shaped by universally relevant human categories of warm and cool{Gibson, 2017 \#9992}.

% compare to humans

\end{abstract}