\begin{abstract}

A characteristic feature of the human mind is the use of concepts \citep{RN18670,RN17954,RN18739}.
To what extent does concept formation require language and culture? 
Here we exploit color as a tool of cognitive science \citep{RN6901,RN18702,RN17999,RN18672} and ask if macaque monkeys have color concepts evident as categories, a question that is unresolved \citep{RN210,RN18502,RN17168}.
Color is continuous yet readily categorized, and for humans, color categories express meaning-laden concepts. 
For example, a ripe strawberry, an embarrassed child, a valentine’s heart, all derive meaning from redness. 
Macaques have the same spectral sensitivity as humans \citep{RN363} and very similar central visual pathways \citep{RN17582}, but for our purposes they usefully lack language. 
We adapted a nonverbal match-to-sample paradigm used in humans to uncover color categories from errors in matches \citep{RN9761} with colors sampling the whole space of colors assumed to be perceptually uniform \citep{RN4684}. 
With this method, macaques seemed to show two consensus color categories, not four as in humans. 
We next determined whether the results were better explained by a model that used cognitive categories or unwitting non-uniformities in the color space. 
The results favor the second explanation, providing evidence of a nonlinear mapping between a true underlying uniform representation of colors and the space used. 
From the empirical results, we reverse-engineered a uniform perceptual space unconfounded by language. 
Although the data provide evidence against a set of consensus color categories shared among monkeys, they do show that individual animals have the potential to form cognitive biases.
The results suggest that human color categories are not innate and imply that the capacity to form categories does not depend on language.

\end{abstract}