\begin{abstract}

A characteristic feature of the human mind is the use of concepts \citep{carey_origin_2000,tenenbaum_how_2011,RN18739}. A long-standing question is the extent to which concept formation requires language. Here we exploit color as a tool of cognitive science \citep{wilken_detection_2004,zhang_discrete_2008,schurgin_psychophysical_2020} and ask if macaque monkeys have color concepts evident as categories. Color is continuous yet readily categorized, and for humans, color categories express meaning-laden ideas. For example, a ripe strawberry, an embarrassed child, a valentine’s heart, all derive meaning from the concept of redness. Macaques and humans have very similar spectral sensitivity \citep{schnapf_spectral_1987, stoughton_psychophysical_2012, gagin_color-detection_2014} and central visual pathways \citep{lafer-sousa_color-biased_2016}, yet for our purposes they usefully lack language. Whether macaques have consensus color categories is unresolved \citep{sandell_color_1979,fagot_cross-species_2006,siuda-krzywicka_biological_2019}, but if they do, then language cannot be required. Alternatively, if macaques do not have color categories, then color categories are probably not innate. We tested macaques by adapting a match-to-sample paradigm used in humans to uncover color categories from errors in matches \citep{bae_why_2015}. When analyzed with the same methods used previously \citep{bae_why_2015,zhang_discrete_2008}, data in macaques showed evidence for two consensus color categories, not four as in humans. Individual macaques showed additional private color categories, indicating that the capacity to form categories does not depend on language. Further analysis with a newer approach \citep{schurgin_psychophysical_2020} shows that the evidence for the two consensus color categories could be explained by an unwitting non-uniformity in the presumed uniform color space, without invoking cognitive mechanisms. These results show that macaques do not have consensus color categories, they suggest that color categories are not innate, and they allow us to reverse-engineer a perceptually uniform color space that is unconfounded by language.  

\end{abstract}