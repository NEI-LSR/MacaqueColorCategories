\begin{abstract}

A characteristic feature of the human mind is the use of concepts \citep{carey_origin_2000,tenenbaum_how_2011,RN18739}. A long-standing question is the extent to which concept formation requires language. Here we exploit color as a tool of cognitive science \citep{wilken_detection_2004,zhang_discrete_2008,schurgin_psychophysical_2020} and ask if macaque monkeys have color concepts evident as categories. Color is continuous yet readily categorized, and for humans, color categories express meaning-laden concepts. For example, a ripe strawberry, an embarrassed child, a valentine’s heart, all derive meaning from the commonly understood concept of redness. Macaques have the same spectral sensitivity as humans \citep{schnapf_spectral_1987} and similar central visual pathways \citep{lafer-sousa_color-biased_2016}, yet for our purposes they usefully lack language. Whether macaques have consensus color categories is unresolved \citep{sandell_color_1979,fagot_cross-species_2006,siuda-krzywicka_biological_2019}, but if they do, then language cannot be required. Alternatively, if macaques do not have color categories, then color categories are probably not innate. We tested macaques with an adapted match-to-sample paradigm used in humans to uncover color categories from errors in matches \citep{bae_why_2015}. When the data were analyzed with the same methods used in humans\citep{bae_why_2015,zhang_discrete_2008}, macaques showed evidence for two consensus color categories, not four as in humans. But further analysis with a newer approach \citep{schurgin_psychophysical_2020} shows that this evidence could be explained an unwitting non-uniformity in the so-called perceptually uniform color space \citep{stockman_colorimetry_2010}, a simpler explanation that does not invoke cognitive mechanisms. The results strongly suggest that color categories are not innate and allow us to reverse-engineer a perceptually uniform color space that is unconfounded by language. Finally, individual macaques did show idiosyncratic private color categories, which implies that the capacity to form categories does not depend on language. 

\end{abstract}