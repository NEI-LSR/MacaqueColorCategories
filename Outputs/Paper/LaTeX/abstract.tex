\begin{abstract}

To what extent does concept formation require language? Here we exploit color to address this question and ask if macaque monkeys have color concepts evident as categories. Macaques have similar cone photoreceptors and central visual circuits to humans, yet they lack language. Whether Old World monkeys such as macaques have consensus color categories is unresolved, but if they do, then language cannot be required; if they do not, then color categories in humans cannot be innate. We tested macaques by adapting a match-to-sample paradigm used in humans to uncover color categories from errors in matches, and analyzed the data using computational simulations that assess the possibility of unrecognized distortions in the perceptual uniformity of color space. The results show that macaques do not have any consensus color categories, and imply that consensus color categories in humans, for which there is ample evidence, must depend upon language.

\end{abstract}