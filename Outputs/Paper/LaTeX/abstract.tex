\begin{abstract}

A feature of the human mind is the use of concepts \citep{RN18743,tenenbaum_how_2011}. One long-standing question is the extent to which concept formation requires language \citep{carey_origin_2000,RN18739}. Here we exploit color as a tool of cognitive science \citep{wilken_detection_2004,zhang_discrete_2008,schurgin_psychophysical_2020} and ask if macaque monkeys have color concepts evident as categories. Color is continuous yet readily categorized, and for humans, color categories express meaning-laden ideas. For example, a ripe strawberry, an embarrassed child, a valentine’s heart, all share the category "red" and derive meaning from that commonly understood concept. Macaques have similar cone photoreceptors and central visual circuits to humans \citep{schnapf_spectral_1987, stoughton_psychophysical_2012, gagin_color-detection_2014,lafer-sousa_color-biased_2016}, yet they lack language. Whether they have consensus color categories is unresolved \citep{sandell_color_1979,fagot_cross-species_2006,RN18699,siuda-krzywicka_biological_2019}, but if they do, then language cannot be required. Alternatively, if macaques do not have color categories, then color categories in humans are probably not innate. We tested macaques by adapting a match-to-sample paradigm used in humans to uncover color categories from errors in matches \citep{bae_why_2015}. Using the same analysis in that work \citep{bae_why_2015,zhang_discrete_2008}, we observed what seemed to be two consensus color categories in macaques, not four as in humans; individual macaques showed additional private color categories, showing that the capacity to form categories does not depend on language. Using computational simulations, we discovered that the two apparent consensus color categories are better explained by an unwitting non-uniformity in the presumed uniform color space than by cognitive mechanisms. The results strongly suggest that macaques do not have any consensus color categories and imply that consensus color categories in humans depend upon language.  

\end{abstract}