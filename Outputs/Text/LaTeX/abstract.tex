\begin{abstract}

Categorical perception is a hallmark of high-level vision, often studied with color to tease apart contributions of perception, cognition, language, and culture. 
Here we ask how macaque monkeys categorize colors.  
Macaques have the same three cone types as humans but lack language and human culture, so their behavior provides useful insight into non-linguistic cognitive and perceptual factors. 
We used a delayed forced-choice (match-to-sample) paradigm that leverages prior work showing a correspondence between memory bias and color categories in humans. 
Macaque color categories inferred with this paradigm do not correspond to basic color categories. 
Instead, the animals show individual differences in how colors are categorized, with all animals showing a consistent set of two categories that correspond roughly to warm (orange-ish) and cool (blue-ish). 
An analysis of errors in the match-to-sample task suggests that the warm category has a cognitive origin while the blue category is perceptual. 


% compare to humans

\end{abstract}