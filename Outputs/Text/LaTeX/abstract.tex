Previous investigations concerning whether macaques use color categories have produced limited and conflicting results. 
The extent to which they do exhibit this behavior sets the extent to which neurophysiological results in macaque can be applied as a model for higher level human visual function, and also allows us to disentangle linguistic / cultural / top-down from innate / biological / earthly / terrestrial theories / factors / drivers of / for why we use color categories.
Using a delayed forced-choice/match-to-sample paradigm, and building on prior work which established the correspondence between memory bias and color categories, we find that the macaques we studied all show evidence of using two shared categories: warm (orange-ish), and cool (blue-ish). 
We find some variability between our animals, with one showing evidence for the use of an additional third category. 
The biases do not appear to align with post-receptoral cardinal mechanisms, though learning rates on the task appear do appear to uncover these mechanisms [to be confirmed].

% compare to humans