Color categorization appears to be a universal human phenomenon. 
Widespread variability in culture-specific color terminology suggests that color categories are linguistic in nature; however, fundamental similarities in color naming and color categorization across languages suggests that there may be some underlying structure which is universally inherent to human cognition and neurophysiology \citep{berlin_basic_1991,gibson_color_2017}. 
Studying categories in non-linguistic animals allows us to pick apart the relative contributions of language and innate factors. It has been previously shown that non-linguistic animals can be trained to perform tasks that involve the categorization of color \citep{sandell_color_1979,fagot_cross-species_2006}. What has been unclear is whether these categories are used without explicit externally-motivated training.

% More about Sandell and/or Fagot?

We developed a method to test for evidence of categorical behavior without explicit categorical training, based on a task which has been extensively used in the investigation of working memory: a delayed forced-choice task. % refs 
In this task, the participant is shown a colored circle on screen, which they remember the color of, and then after a delay, they select a circle of matching color from a set of differently colored circles.

This task has traditionally been used in working memory experiments because color was seen as a simple continuous scale with well-defined perceptual uniformity across the scale. 
Unfortunately, for working memory researchers, these assumptions have been shown to be ill-founded - \cite{bae_why_2015} found that certain colors were remembered more accurately than others and that responses for certain colors were biased towards other colors.
The pattern in responses could be accounted for by a model that encoded a memorized color in two distinct ways simultaneously - as a point on a continuous scale and \emph{also} as a member of one of a number of categories that carved up colorspace. 
For humans they found that a four-category model fit their data well.

In seeking to explain this result, they showed that these categories correspond relatively well with categories identified in a separate experiment (with separate observers) to the categories recovered from people's linguistic categories for color. 
This shows us that, at least at the population level, there is an apparent correspondence between the location of color category centers and the biases in response.
It is possible to train macaques on a delayed forced-choice task such as this, thus we are able to use such a task, alongside the theoretical constructs of \cite{bae_why_2015}, to investigate innate use of color categories in macaques.


% Summary of things to add
% - TCC