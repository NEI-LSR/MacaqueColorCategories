One popular theory to explain the existance of color categories is to enable efficient inter-person communication via language; reducing the essentially infinate range of perceivable colors down to a subset of categories allows us communicate that information quickly and efficently to another person via spoken language.

Other theories exist.

We recently found that a non-linguistic animal (macaques) appear to naturally use color categories. This strikes a blow to the idea that color categories exist \emph{solely} for linguistic inter-personal communication.