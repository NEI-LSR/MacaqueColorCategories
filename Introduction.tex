Color categorization appears to be a universal human phenomenon. 
Widespread variability in culture-specific color terminology suggests that color categories are linguistic in nature; however, fundamental similarities in color naming and color categorization across languages suggests that there may be some underlying structure which is universally inherent to human cognition and neurophysiology \citep{berlin_basic_1991,gibson_color_2017}. 

It has been previously shown that non-linguistic animals can be trained to perform tasks that involve the categorization of color \citep{sandell_color_1979,fagot_cross-species_2006}. 
What has been unclear is whether these categories are used without explicit externally-motivated training.

% More about Sandell and/or Fagot?

We developed a method to test for evidence of categorical behavior without explicit categorical training, based on a task which has been extensively used in the investigation of working memory. % refs 
In this task, the participant is shown a colored circle on screen, which they remember the color of, and then after a delay, they select a circle of matching color from a set of differently colored circles.

This task has traditionally been used in working memory experiments because color was seen as a simple continuous scale with well-defined perceptual uniformity across the scale. 
Unfortunately, for working memory researchers, these assumptions have been shown to be ill-founded.
\cite{bae_why_2015} found that certain colors were remembered more accurately than others and that responses for certain colors were biased towards other colors.
The pattern in responses could be accounted for by a model that encoded a memorized color in two distinct ways simultaneously - as a point on a continuous scale and \emph{also} as a member of one of a number of categories that carved up colorspace. 
For humans they found that a four-category model performed as well as models with higher numbers of categories (though it is possible this was due to a noise floor in their data).

In seeking to explain this result, they showed that these categories correspond relatively well with categories identified in a separate experiment (with separate observers) to the categories recovered from people's linguistic categories for color. 
As far as we are aware there is no data that shows correspondence at an individual level between linguistic category definitions and data from an experiment such as this, but the \cite{bae_why_2015} data show us that at least at the population level there is an apparent correspondence between the location of color category centers and the biases in response.

\cite{panichello_error-correcting_2019} extended this line of research in several key ways. 
Firstly, they extended the logic to account for the fact that when the memory period is extended, participants seemed to draw more heavily on their categorical representation (you can test this yourself - how well can you recall the color of an item you saw fleetingly yesterday? 
You can probably only confidently report the color category). 
They did so by casting the category centers as `attractor points' in the space - over time noise would be added to the continuous representation and this noise would be biased by a drift function which would cause the memorized color to drift towards the closest attractor point over time. 
This provides a computational rationale for why such a mechanism would have value - it places an upper bound on the amount of degradation that can occur as a result of noise acting upon a memory; it can only drift to the nearest attractor point, after which it gets `stuck'.

As well as collecting data from a large number of humans on this task, \cite{panichello_error-correcting_2019} also collected data from 2 macaques on a related task. 
These animals both showed behavior that was consistent with a model that included attractor dynamics, but the results of each animal didn't show clear correspondence with the other, or with the human data.
It is unclear whether the comparison with the human data is valid, considering a number of differences between the task presented to the humans and the task presented to the macaques.

Here we present the results of a task where 3 macaques performed a task which could also be administered in humans.